\section{Git}
\label{sec:git}

Introduction

Why use GIT?


Git is a mature, actively maintained open source project originally developed in 2005 by Linus Torvalds, the famous creator of the Linux operating system kernel. A staggering number of software projects rely on Git for version control, including commercial projects as well as open source. Developers who have worked with Git are well represented in the pool of available software development talent and it works well on a wide range of operating systems and \glspl{ide}.

Having a distributed architecture, Git is an example of a \gls{dvcs}. Rather than have only one single place for the full version history of the software as is common in once-popular version control systems like \gls{vcs} or \gls{svn}, in Git, every developer's working copy of the code is also a repository that can contain the full history of all changes.

In addition to being distributed, Git has been designed with performance, security and flexibility in mind.

The raw performance characteristics of Git are very strong when compared to many alternatives. Committing new changes, branching, merging and comparing past versions are all optimized for performance. The algorithms implemented inside Git take advantage of deep knowledge about common attributes of real source code file trees, how they are usually modified over time and what the access patterns are.

Git has been designed with the integrity of managed source code as a top priority. The content of the files as well as the true relationships between files and directories, versions, tags and commits, all of these objects in the Git repository are secured with a cryptographically secure hashing algorithm called SHA1. This protects the code and the change history against both accidental and malicious change and ensures that the history is fully traceable.

One of Git's key design objectives is flexibility. Git is flexible in several respects: in support for various kinds of nonlinear development workflows, in its efficiency in both small and large projects and in its compatibility with many existing systems and protocols.